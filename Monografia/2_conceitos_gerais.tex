\section{CONCEITOS GERAIS E REVISÃO DA LITERATURA}

%Lembre-se que as sessões e sub-sessões são determinadas por si para adequar-se ao seu trabalho.

Neste capítulo deve ser proporcionado o estado da arte / referencial teórico sobre o tema a que se refere o estudo. Um bom pesquisador não deve repetir trabalhos já concluídos ou que já estão em andamento. Por isso esta sessão é onde o autor demonstra até onde vai a pesquisa atual no campo de estudos em questão e estabelece as bases sobre as quais desenvolverá o estudo proposto. 

A seguir são mostrados alguns exemplos de como deve-se inserir as figuras e tabelas. A Figura 01 mostra um exemplo de como inserir uma figura no texto. A Tabela 01 mostra o exemplo de como uma tabela deve ser inserida.

\begin{figure}[h]
\centering
\includegraphics[height=6.2cm]{imagens/keep}
\caption{Exemplo de como inserir Figura}
\label{fig:exemplo}
\end{figure}

\begin{table}[h]
\centering
\caption{ Modelo de como as tabelas devem ser inseridas no texto }
\vspace{0.2in}
\newcolumntype{C}{>{\centering\arraybackslash}X}%
\newcommand{\rowstyle}[1]{%
  \protected\gdef\currentrowstyle{#1}%
}
\begin{tabularx}{\textwidth}{>{\bf}C|C|C|C}
\hline 
\textbf {Índice} & \textbf{Coluna 01} &\textbf{ Coluna 02} & \textbf{Coluna 03} \\ \hline \hline
Linha 01 & & & \\ \hline
Linha 02 & & & \\ \hline                         

\end{tabularx}
\end{table}
